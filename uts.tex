\documentclass{article}
\usepackage{hyperref}
\usepackage{listings}
\usepackage{xcolor}

\title{StructArray}
\author{}
\date{}

\begin{document}

\maketitle

\section*{Purpose}
The purpose of this demo is so we don't need to define a lot of variables with the type \texttt{struct} that we need for certain functions.

\section*{Reason for OS}
The reason for using an OS is so we don't need to define and create our own scheduler, thread model, mutex, etc.

\section*{Example Source Code}

\begin{lstlisting}[language=C++,basicstyle=\ttfamily\footnotesize,keywordstyle=\color{blue},commentstyle=\color{gray},stringstyle=\color{red}]
/*
 * Example of a basic FreeRTOS queue
 * https://www.freertos.org/Embedded-RTOS-Queues.html
 * src : https://wokwi.com/projects/new/arduino-uno
 */

// Include Arduino FreeRTOS library
#include <Arduino_FreeRTOS.h>

// Include queue support
#include <queue.h>

// Define a Structure Array
struct Arduino{
  int pin[2];
  int ReadValue[2];
};

//Function Declaration
void Blink(void *pvParameters);
void POT(void *pvParameters);
void TaskSerial(void *pvParameters);

/* 
 * Declaring a global variable of type QueueHandle_t 
 * 
 */
QueueHandle_t structArrayQueue;

void setup() {

  /**
   * Create a queue.
   * https://www.freertos.org/a00116.html
   */
structArrayQueue=xQueueCreate(10, //Queue length
                              sizeof(struct Arduino)); //Queue item size
                              
if(structArrayQueue!=NULL){

  
  xTaskCreate(TaskBlink,  // Task function
              "Blink",// Task name
              128,// Stack size 
              NULL,
              0,// Priority
              NULL);
 
 // Create other task that publish data in the queue if it was created.
  xTaskCreate(POT,// Task function
              "AnalogRead",// Task name
              128,  // Stack size
              NULL,
              2,// Priority
              NULL);

   // Create task that consumes the queue if it was created.
   xTaskCreate(TaskSerial,// Task function
              "PrintSerial",// A name just for humans
              128,// This stack size can be checked & adjusted by reading the Stack Highwater
              NULL,
              1, // Priority, with 3 (configMAX_PRIORITIES - 1) being the highest, and 0 being the lowest.
              NULL);
}
}

void loop() {}

/* 
 * Blink task. 
 * See Blink_AnalogRead example. 
 */
void TaskBlink(void *pvParameters){
  (void) pvParameters;
  
  pinMode(13,OUTPUT);
  
  digitalWrite(13,LOW);
  
  for(;;)
  {
    digitalWrite(13,HIGH);
    vTaskDelay(250/portTICK_PERIOD_MS);
    digitalWrite(13,LOW);
    vTaskDelay(250/portTICK_PERIOD_MS);
  }
}

/**
 * Analog read task for Pin A0 and A1
 * Reads an analog input on pin 0 and pin 1 
 * Send the readed value through the queue.
 * See Blink_AnalogRead example.
 */
void POT(void *pvParameters){
  (void) pvParameters;
  pinMode(A0,INPUT);
  pinMode(A1,INPUT);
  for (;;){
  // Read the input on analog pin 0:
  struct Arduino currentVariable;
  currentVariable.pin[0]=0;
  currentVariable.pin[1]=1;
  currentVariable.ReadValue[0]=analogRead(A0);
  currentVariable.ReadValue[1]=analogRead(A1);  

  /**
    * Post an item on a queue.
    * https://www.freertos.org/a00117.html
    */
  xQueueSend(structArrayQueue,&currentVariable,portMAX_DELAY);

  // One tick delay (15ms) in between reads for stability
  vTaskDelay(1);
  }
}

/**
 * Serial task.
 * Prints the received items from the queue to the serial monitor.
 */
void TaskSerial(void *pvParameters){
  (void) pvParameters;

  // Init Arduino serial
  Serial.begin(9600);

  // Wait for serial port to connect. Needed for native USB, on LEONARDO, MICRO, YUN, and other 32u4 based boards.
  while (!Serial) {
    vTaskDelay(1)
  }
  
  for (;;){
    
    struct Arduino currentVariable;

   /**
     * Read an item from a queue.
     * https://www.freertos.org/a00118.html
     */
    if(xQueueReceive(structArrayQueue,&currentVariable,portMAX_DELAY) == pdPASS ){
      for(int i=0;i<2;i++){
      Serial.print("PIN:");
      Serial.println(currentVariable.pin[i]);
      Serial.print("value:");
      Serial.println(currentVariable.ReadValue[i]);
      }
    }
    vTaskDelay(500/portTICK_PERIOD_MS);
  }
}

\end{lstlisting}

\section*{Explanation}

\begin{itemize}
    \item The new data structure is defined at this line with the member \texttt{pin} and \texttt{ReadValue} as an array of integers with a size of 2.
\begin{lstlisting}[language=C++,basicstyle=\ttfamily\footnotesize]
struct Arduino{
  int pin[2];
  int ReadValue[2];
};
\end{lstlisting}

    \item This line will create the handler for the queue.
\begin{lstlisting}[language=C++,basicstyle=\ttfamily\footnotesize]
QueueHandle_t structArrayQueue;
\end{lstlisting}

    \item The queue will be created using the \texttt{xQueueCreate} function that accepts the length of the queue and the size of the data that will be put inside the queue.
\begin{lstlisting}[language=C++,basicstyle=\ttfamily\footnotesize]
structArrayQueue=xQueueCreate(10, sizeof(struct Arduino));
\end{lstlisting}

    \item The first function that will be launched with the thread is \texttt{TaskBlink} that accepts a pointer of void as the argument.
\begin{lstlisting}[language=C++,basicstyle=\ttfamily\footnotesize]
void TaskBlink(void *pvParameters){
  (void) pvParameters;
  
  pinMode(13,OUTPUT);
  
  digitalWrite(13,LOW);
  
  for(;;)
  {
    digitalWrite(13,HIGH);
    vTaskDelay(250/portTICK_PERIOD_MS);
    digitalWrite(13,LOW);
    vTaskDelay(250/portTICK_PERIOD_MS);
  }
}
\end{lstlisting}

    \item \texttt{pinMode(13, OUTPUT)} sets the 13th pin to be output.
    \item \texttt{digitalWrite(13, LOW)} will give low electricity to pin 13.
    \item \texttt{digitalWrite(13, HIGH)} will give high electricity to pin 13.

    \item The second function that will be launched with the second task is \texttt{POT}.
\begin{lstlisting}[language=C++,basicstyle=\ttfamily\footnotesize]
void POT(void *pvParameters){
  (void) pvParameters;
  pinMode(A0,INPUT);
  pinMode(A1,INPUT);
  for (;;){
  // Read the input on analog pin 0:
  struct Arduino currentVariable;
  currentVariable.pin[0]=0;
  currentVariable.pin[1]=1;
  currentVariable.ReadValue[0]=analogRead(A0);
  currentVariable.ReadValue[1]=analogRead(A1);  

  /**
    * Post an item on a queue.
    * https://www.freertos.org/a00117.html
    */
  xQueueSend(structArrayQueue,&currentVariable,portMAX_DELAY);

  // One tick delay (15ms) in between reads for stability
  vTaskDelay(1);
  }
}
\end{lstlisting}

    \item \texttt{xQueueSend()} accepts the handler, the data, and lastly the delay. The delay has a maximum amount of time the task should block waiting for space to become available on the queue, should it already be full. The call will return immediately if the queue is full and \texttt{xTicksToWait} is set to 0. The time is defined in tick periods so the constant \texttt{portTICK\_PERIOD\_MS} should be used to convert to real time if this is required. 

    \item The last task that will be spawned is \texttt{TaskSerial}.
\begin{lstlisting}[language=C++,basicstyle=\ttfamily\footnotesize]
void TaskSerial(void *pvParameters){
  (void) pvParameters;

  // Init Arduino serial
  Serial.begin(9600);

  // Wait for serial port to connect. Needed for native USB, on LEONARDO, MICRO, YUN, and other 32u4 based boards.
  while (!Serial) {
    vTaskDelay(1);
  }
  
  for (;;){
    
    struct Arduino currentVariable;

   /**
     * Read an item from a queue.
     * https://www.freertos.org/a00118.html
     */
    if(xQueueReceive(structArrayQueue,&currentVariable,portMAX_DELAY) == pdPASS ){
      for(int i=0;i<2;i++){
      Serial.print("PIN:");
      Serial.println(currentVariable.pin[i]);
      Serial.print("value:");
      Serial.println(currentVariable.ReadValue[i]);
      }
    }
    vTaskDelay(500/portTICK_PERIOD_MS);
  }
}
\end{lstlisting}

    \item \texttt{Serial.begin()} will initialize the lowest serial frequency.
    \item \texttt{xQueueReceive()} accepts the handler, the data, and lastly the delay. The function will return \texttt{pdPASS} or \texttt{pdFAIL}.

    \item The array size can be changed. In the example, it has a length of 2.
\begin{lstlisting}[language=C++,basicstyle=\ttfamily\footnotesize]
#define n 10;
struct Arduino{
  int pin[n];
  int ReadValue[n];
};
\end{lstlisting}

    \item With that code, the length of the array will be 10.

    \item About \texttt{xTaskCreate(function, readable function name, stack size, args to func, priority, handler for the task)}:
    \begin{itemize}
        \item From the example, if the \texttt{xTaskCreate} priority argument is set to be bigger, it will have the most priority and get to run first.
        \item Priority is \texttt{uint8\_t} with the alias of \texttt{UBaseType\_t}.
        \item Any unsigned 8-bit integer is possible as an argument for the \texttt{xTaskCreate} function.
        \item When the priority is 0, it will be an idle task that will run only when idle (no other work to do).
    \end{itemize}

    \item About \texttt{vTaskDelay(how long the task will be delayed in ticks)}:
    \begin{itemize}
        \item \texttt{vTaskDelay()} specifies a time at which the task wishes to unblock relative to the time at which \texttt{vTaskDelay()} is called. For example, specifying a block period of 100 ticks will cause the task to unblock 100 ticks after \texttt{vTaskDelay()} is called.
    \end{itemize}

\end{itemize}

\end{document}
